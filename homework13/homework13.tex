%%%%%%%%%%%%%%%%%%%%%%%%%%%%%%%%%%%%%%%%%%%%%%%%%%%%%%%%%%%%%%%
%
% Welcome to writeLaTeX --- just edit your LaTeX on the left,
% and we'll compile it for you on the right. If you give
% someone the link to this page, they can edit at the same
% time. See the help menu above for more info. Enjoy!
%
%%%%%%%%%%%%%%%%%%%%%%%%%%%%%%%%%%%%%%%%%%%%%%%%%%%%%%%%%%%%%%%

% --------------------------------------------------------------
% This is all preamble stuff that you don't have to worry about.
% Head down to where it says "Start here"
% --------------------------------------------------------------
 
\documentclass[12pt]{article}
 
\usepackage[margin=1in]{geometry}
\usepackage{amsmath,amsthm,amssymb}

\usepackage{listings}
\usepackage{xcolor}

\usepackage{tikz}
\usetikzlibrary{shapes,positioning}

\tikzset{ell/.style={circle,draw,minimum height=0.5cm,minimum width=0.5cm,inner sep=0.2cm}}
\tikzset{rec/.style={rectangle,draw,minimum height=0.5cm,minimum width=0.5cm,inner sep=0.2cm}}

%New colors defined below
\definecolor{codegreen}{rgb}{0,0.6,0}
\definecolor{codegray}{rgb}{0.5,0.5,0.5}
\definecolor{codepurple}{rgb}{0.58,0,0.82}
\definecolor{backcolour}{rgb}{0.95,0.95,0.92}

%Code listing style named "mystyle"
\lstdefinestyle{mystyle}{
  backgroundcolor=\color{backcolour}, commentstyle=\color{codegreen},
  keywordstyle=\color{magenta},
  numberstyle=\tiny\color{codegray},
  stringstyle=\color{codepurple},
  basicstyle=\ttfamily\footnotesize,
  breakatwhitespace=false,         
  breaklines=true,                 
  captionpos=b,                    
  keepspaces=true,                 
  numbers=left,                    
  numbersep=5pt,                  
  showspaces=false,                
  showstringspaces=false,
  showtabs=false,                  
  tabsize=2
}

%"mystyle" code listing set
\lstset{style=mystyle}

 
\newcommand{\N}{\mathbb{N}}
\newcommand{\Z}{\mathbb{Z}}
 
\newenvironment{theorem}[2][Theorem]{\begin{trivlist}
\item[\hskip \labelsep {\bfseries #1}\hskip \labelsep {\bfseries #2.}]}{\end{trivlist}}
\newenvironment{lemma}[2][Lemma]{\begin{trivlist}
\item[\hskip \labelsep {\bfseries #1}\hskip \labelsep {\bfseries #2.}]}{\end{trivlist}}
\newenvironment{exercise}[2][Exercise]{\begin{trivlist}
\item[\hskip \labelsep {\bfseries #1}\hskip \labelsep {\bfseries #2.}]}{\end{trivlist}}
\newenvironment{problem}[2][Problem]{\begin{trivlist}
\item[\hskip \labelsep {\bfseries #1}\hskip \labelsep {\bfseries #2.}]}{\end{trivlist}}
\newenvironment{question}[2][Question]{\begin{trivlist}
\item[\hskip \labelsep {\bfseries #1}\hskip \labelsep {\bfseries #2.}]}{\end{trivlist}}
\newenvironment{corollary}[2][Corollary]{\begin{trivlist}
\item[\hskip \labelsep {\bfseries #1}\hskip \labelsep {\bfseries #2.}]}{\end{trivlist}}

\newenvironment{solution}{\begin{proof}[Solution]}{\end{proof}}
 
\begin{document}
 
% --------------------------------------------------------------
%                         Start here
% --------------------------------------------------------------
 
\title{Homework 13}%replace X with the appropriate number
\author{Mengxiang Jiang\\ %replace with your name
CSEN 5303 Foundations of Computer Science} %if necessary, replace with your course title
 
\maketitle

\begin{problem}{1}
    Determine which rule of inference is used in each of the following arguments.
    \begin{quote}
        1. Kangaroos live in Australia and are marsupials. Therefore, kangaroos are marsupials.\\
        Simplification: $(p \land q) \rightarrow p$\\\\
        2. It is hotter than 100 degrees today or the pollution is dangerous. It is less than 100 degrees
        outside today. Therefore, the pollution is dangerous.\\
        Disjunctive syllogism: $[(p \lor q) \land \neg p] \rightarrow q$\\\\
        3. Linda is an excellent swimmer. If Linda is an excellent swimmer, then she can work as a
        lifeguard. Therefore, Linda can work as a lifeguard.\\
        Modus ponens: $[p \land (p \rightarrow q)] \rightarrow q$\\\\
        4. Steve will work at a computer company this summer. Therefore, this summer Steve will
        work at a computer company or he will be in beach bun.\\
        Addition: $p \rightarrow (p \lor q)$\\\\
        5. If I work all night on this homework, then I can answer all the exercises. If I answer all the
        exercises, I will understand the material. Therefore, if I work all night on this homework,
        then I will understand the material.\\
        Hypothetical syllogism: $[(p \rightarrow q) \land (q \rightarrow r)] \rightarrow (p \rightarrow r)$\\\\
        6. If George does not have eight legs, then he is not an insect. George is an insect. Therefore,
        George has eight legs.\\
        Modus tollens: $[\neg q \land (p \rightarrow q)] \rightarrow \neg p$
    \end{quote}
\end{problem}
\pagebreak
\begin{problem}{2}
    Let P($n$) be the statement that $1^2 + 2^2 + \ldots + n^2 = n(n + 1)(2n + 1)/6$ for the positive integer $n$.
    \begin{quote}
        1. What is the statement P(1)?\\
        $1^2 = 1(1+1)(2+1)/6$\\\\
        2. Show that P(1) is ture, completing the basis step of the proof.\\
        $1^2 = 1(1) = 1$\\ 
        $1(1+1)(2+1)/6 = 1(2)(3)/6 = 6/6 = 1$\\
        $\therefore$ P(1) is true.\\\\
        3. What is the inductive hypothesis?\\
        Assume P($k$) is true for some integer $k \geq 1$.\\\\
        4. What do you need to prove in the inductive step?\\
        Prove P($k+1$) is true using P($k$).\\\\
        5. Complete the inductive step.\\
        P($k+1$): $1^2 + 2^2 + \ldots + k^2 + (k+1)^2 = (1^2 + 2^2 + \ldots + k^2) + (k+1)^2$\\
        $$=\frac{k(k+1)(2k+1)}{6}+(k+1)^2 = (k+1)\left(\frac{2k^2+k}{6}+k+1\right)$$
        $$=(k+1)\left(\frac{2k^2+7k+6}{6}\right) = (k+1)\left(\frac{(k+2)(2k+3)}{6}\right)$$
        $$=\frac{\Bigl(k+1\Bigr)\Bigl((k+1)+1\Bigr)\Bigl(2(k+1)+1\Bigr)}{6}$$
        6. Explain why these steps show that this formula is true whenever n is a positive integer.\\
        Since we have shown P(1) is true in step 2, we can let $k=1$ in step 3 and show P(2) is true in step 5.
        We can then set $k=2$ in step 3 and show P(3) is true in step 5, and so on for all the integers. 
    \end{quote}
\end{problem}
\pagebreak
\begin{problem}{6}
    Prove each of the following statements. 
    \begin{quote}
        1. 2 divides $n^2 + n$ whenever $n$ is a positive integer.\\
        Two cases for $n$:\\
        Case 1: $ n \equiv 0 \pmod{2} \rightarrow n^2 + n \equiv 0^2 + 0 \equiv 0 \pmod{2} \rightarrow 2|(n^2 + n)$\\
        Case 2: $ n \equiv 1\pmod{2} \rightarrow n^2 +n \equiv 1^2 + 1 \equiv 2 \equiv 0 \pmod{2} \rightarrow  2|(n^2 + n)$\\\\
        2. 3 divides $n^3+2n$ whenever $n$ is a positive integer.\\
        Three cases for $n$:\\
        Case 1: $n \equiv 0 \pmod{3} \rightarrow n^3 + 2n \equiv 0^3 + 2(0) \equiv 0 \pmod{3} \rightarrow 3|(n^3+2n)$\\ 
        Case 2: $n \equiv 1 \pmod{3} \rightarrow n^3 + 2n \equiv 1^3 + 2(1) \equiv 3 \equiv 0 \pmod{3} \rightarrow 3|(n^3+2n)$\\
        Case 3: $n \equiv 2 \pmod{3} \rightarrow n^3 + 2n \equiv 2^3 + 2(2) \equiv 12 \equiv 0 \pmod{3} \rightarrow 3|(n^3+2n)$\\\\
        3. 5 divides $n^5 - n$ whenever n is a nonnegative integer.\\
        Five cases for $n$:\\
        Case 1: $n \equiv 0 \pmod{5} \rightarrow n^5 - n \equiv 0^5 - 0 \equiv 0 \pmod{5} \rightarrow 5|(n^5-n)$\\
        Case 2: $n \equiv 1 \pmod{5} \rightarrow n^5 - n \equiv 1^5 - 1 \equiv 0 \pmod{5} \rightarrow 5|(n^5-n)$\\
        Case 3: $n \equiv 2 \pmod{5} \rightarrow n^5 - n \equiv 2^5 - 2 \equiv 30 \equiv 0 \pmod{5} \rightarrow 5|(n^5-n)$\\
        Case 4: $n \equiv 3 \pmod{5} \rightarrow n^5 - n \equiv n(n^4 - 1) \equiv n(n^2-1)(n^2+1)$\\
        $\equiv 3(3^2-1)(3^2+1) \equiv 3(8)(10) \equiv 3(8)(0) \equiv 0 \pmod{5} \rightarrow 5|(n^5-n)$\\
        Case 5: $n \equiv 4 \pmod{5} \rightarrow n^5 - n \equiv n(n^4 - 1) \equiv n(n^2-1)(n^2+1)$\\
        $\equiv n(n+1)(n-1)(n^2+1) \equiv 4(0)(3)(4^2+1) \equiv 0 \pmod{5} \rightarrow 5|(n^5-n)$\\\\
        4. 6 divides $n^3 - n$ whenever $n$ is a nonnegative integer.\\
        $n^3 - n = n(n^2-1) = n(n+1)(n-1) = (n-1)(n)(n+1)$\\
        $ \therefore n^3 - n$ is the product of 3 consecutive integers.\\
        Given 3 consecutive integers, at least one must be a multiple of 3, and one (could be the same one) a multiple of 2 (pigeonhole principle).
        The product of the 3 integers therefore contains both 2 and 3 as factors, therefore the product must be divisible by 6.\\

    \end{quote}
\end{problem}
% --------------------------------------------------------------
%     You don't have to mess with anything below this line.
% --------------------------------------------------------------
 
\end{document}