%%%%%%%%%%%%%%%%%%%%%%%%%%%%%%%%%%%%%%%%%%%%%%%%%%%%%%%%%%%%%%%
%
% Welcome to writeLaTeX --- just edit your LaTeX on the left,
% and we'll compile it for you on the right. If you give
% someone the link to this page, they can edit at the same
% time. See the help menu above for more info. Enjoy!
%
%%%%%%%%%%%%%%%%%%%%%%%%%%%%%%%%%%%%%%%%%%%%%%%%%%%%%%%%%%%%%%%

% --------------------------------------------------------------
% This is all preamble stuff that you don't have to worry about.
% Head down to where it says "Start here"
% --------------------------------------------------------------
 
\documentclass[12pt]{article}
 
\usepackage[margin=1in]{geometry}
\usepackage{amsmath,amsthm,amssymb}

\usepackage{listings}
\usepackage{xcolor}

%New colors defined below
\definecolor{codegreen}{rgb}{0,0.6,0}
\definecolor{codegray}{rgb}{0.5,0.5,0.5}
\definecolor{codepurple}{rgb}{0.58,0,0.82}
\definecolor{backcolour}{rgb}{0.95,0.95,0.92}

%Code listing style named "mystyle"
\lstdefinestyle{mystyle}{
  backgroundcolor=\color{backcolour}, commentstyle=\color{codegreen},
  keywordstyle=\color{magenta},
  numberstyle=\tiny\color{codegray},
  stringstyle=\color{codepurple},
  basicstyle=\ttfamily\footnotesize,
  breakatwhitespace=false,         
  breaklines=true,                 
  captionpos=b,                    
  keepspaces=true,                 
  numbers=left,                    
  numbersep=5pt,                  
  showspaces=false,                
  showstringspaces=false,
  showtabs=false,                  
  tabsize=2
}

%"mystyle" code listing set
\lstset{style=mystyle}

 
\newcommand{\N}{\mathbb{N}}
\newcommand{\Z}{\mathbb{Z}}
 
\newenvironment{theorem}[2][Theorem]{\begin{trivlist}
\item[\hskip \labelsep {\bfseries #1}\hskip \labelsep {\bfseries #2.}]}{\end{trivlist}}
\newenvironment{lemma}[2][Lemma]{\begin{trivlist}
\item[\hskip \labelsep {\bfseries #1}\hskip \labelsep {\bfseries #2.}]}{\end{trivlist}}
\newenvironment{exercise}[2][Exercise]{\begin{trivlist}
\item[\hskip \labelsep {\bfseries #1}\hskip \labelsep {\bfseries #2.}]}{\end{trivlist}}
\newenvironment{problem}[2][Problem]{\begin{trivlist}
\item[\hskip \labelsep {\bfseries #1}\hskip \labelsep {\bfseries #2.}]}{\end{trivlist}}
\newenvironment{question}[2][Question]{\begin{trivlist}
\item[\hskip \labelsep {\bfseries #1}\hskip \labelsep {\bfseries #2.}]}{\end{trivlist}}
\newenvironment{corollary}[2][Corollary]{\begin{trivlist}
\item[\hskip \labelsep {\bfseries #1}\hskip \labelsep {\bfseries #2.}]}{\end{trivlist}}

\newenvironment{solution}{\begin{proof}[Solution]}{\end{proof}}
 
\begin{document}
 
% --------------------------------------------------------------
%                         Start here
% --------------------------------------------------------------
 
\title{Homework 5}%replace X with the appropriate number
\author{Mengxiang Jiang\\ %replace with your name
CSEN 5303 Foundations of Computer Science} %if necessary, replace with your course title
 
\maketitle
 
\begin{problem}{1 (Stacks)} %You can use theorem, exercise, problem, or question here.  Modify x.yz to be whatever number you are proving
Let S1 and S2 be two stacks.\\
1. Is it possible to keep two stacks in a single array, if one grows from position 1 of the array, and the other grows from the last position? \\
2. Write a procedure Push(x, S) that pushes element x onto stack S, where S is one or the other of these two stacks. Include all necessary error checks in your procedure.\\

1. Yes, as long as the two stacks don't grow to occupy a bigger size than the array when added together, then there shouldn't be a problem.

2. \begin{verbatim}
    procedure Push(x: item, S: Stack);
        var
            head_idx, tail_idx: integer;
            head_type: boolean;
        begin
            head_idx := S.head_idx;
            tail_idx := S.tail_idx;
            if head_idx + 1 >= tail_idx then
                throw StackFullError;
            else
                begin
                    if head_type = true then
                        begin
                            S.head_idx := head_idx + 1;
                            S.array[S.head_idx] := x;
                        end;
                    else
                        begin
                            S.tail_idx := tail_idx - 1;
                            S.array[S.tail_idx] := x;
                        end;
                end;
        end;
\end{verbatim}

\end{problem}

\begin{problem}{2 (Stacks)}
    Consider the fundamental theorem of arithmetic, which is stated as follows: 
    Every positive integer greater than 1 can be written uniquely as a prime or as the product of two or more primes,
    where the prime factors are written in order of nondecreasing size. 
    We want to use a stack to read a number and print all of its prime divisors in descending order.
    For example, with the integer 2100, the output should be:\\
    \begin{center}
        7 5 5 3 2 2
    \end{center}

    1. Write an algorithm, called \emph{Prime\_Factorization}, which accepts a positive integer greater than 1,
    and generates its prime factorization according to the above-mentioned theorem. [\emph{Hint:} The smallest divisor greater than 1 of any integer is guaranteed to be a prime.]
\begin{verbatim}
    function Prime_Factorization(n: integer) : StackInt;
        var
            factors : StackInt;
            p : integer;
        begin
            p := 2;
            while n > 1 do
                begin
                    if n mod p = 0 then
                        begin
                            factors.push(p);
                            n := n / p;
                        end;
                    else
                        p := p + 1;
                end;
            return factors;
        end;
\end{verbatim}

    2. Propose a stack class to accommodate this prime decomposition.
    It should have at least two member functions: one to compute the prime factorization
    of an integer, and one to print all corresponding prime divisors in descending order.
\begin{verbatim}
    class PrimeFactorizationStack
        private factors_stack
        private n

        public procedure new(n)
        // initialize variables of PrimeFactorizationStack

        private procedure prime_factorization()
        // fill the factors_stack with the prime factors

        public procedure print_prime_factors()
        // call prime_decomposition then use the stack to print the factors
\end{verbatim}

    3. Give an implementation of all member functions defined in the above stack class.
\begin{verbatim}
    procedure prime_factorization();
        var
            n, p : integer;
        begin
            if self.factors_stack.is_empty() = true then
                begin
                    self.factors_stack = new StackInt;
                    n := self.n;
                    p := 2;
                    while n > 1 do
                        begin
                            if n mod p = 0 then
                                begin
                                    factors.push(p);
                                    n := n / p;
                                end;
                            else
                                p := p + 1;
                        end;
                end;
        end;
    
    procedure print_prime_factors();
        var
            p : integer;
        begin
            if self.factors_stack.is_empty() = true then 
                self.prime_factorization();
            while (self.factors_stack.is_empty() = false) do
                begin
                    p := self.factors_stack.top();
                    print(p.to_str() + " ");
                    self.factors_stack.pop();
                end
        end
\end{verbatim}
\end{problem}

\pagebreak

\begin{problem}{3 (Stacks)}
    Write a segment of code to perform each of the following operations. You
    may call any of the member functions of Stack Type. The details of the stack type are encapsulated;
    you may use only the stack operations in the specification to perform the operations. (You may
    declare additional stack objects). 

    a. Set $secondElement$ to the second element in the stack, leaving the stack without its original top two elements.
\begin{verbatim}
    //assuming we are given s1 as the stack
    s1.pop();
    ItemType secondElement = s1.top();
    s1.pop();
\end{verbatim}  

    b. Set $bottom$ equal to the bottom element in the stack, leaving the stack empty.
\begin{verbatim}
    //assuming we are given s1 as the stack
    ItemType bottom;
    while (s1.isEmpty() == false) {
        bottom = s1.top();
        s1.pop();
    }
\end{verbatim}    

    c. Set $bottom$ equal to the bottom element in the stack, leaving the stack unchanged.
\begin{verbatim}
    //assuming we are given s1 as the stack
    ItemType bottom;
    StackType s2 = new StackType();
    while (s1.isEmpty() == false) {
        bottom = s1.top();
        s1.pop();
        s2.push(bottom);
    }
    while (s2.isEmpty() == false) {
        s1.push(s2.top());
        s2.pop();
    }
\end{verbatim}

\pagebreak
    d. Make a copy of the stack, leaving the stack unchanged.
\begin{verbatim}
    //assuming we are given s1 as the stack
    ItemType temp;
    //s3 is the copy, s2 is used to help
    StackType s2 = new StackType();
    StackType s3 = new StackType();
    while (s1.isEmpty() == false) {
        s2.push(s1.top());
        s1.pop();
    }
    while (s2.isEmpty() == false) {
        temp = s2.top()
        s1.push(temp);
        s3.push(temp);
        s2.pop();
    }
\end{verbatim}
\end{problem}
 
% --------------------------------------------------------------
%     You don't have to mess with anything below this line.
% --------------------------------------------------------------
 
\end{document}