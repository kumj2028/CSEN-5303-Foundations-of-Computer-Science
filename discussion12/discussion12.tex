%%%%%%%%%%%%%%%%%%%%%%%%%%%%%%%%%%%%%%%%%%%%%%%%%%%%%%%%%%%%%%%
%
% Welcome to writeLaTeX --- just edit your LaTeX on the left,
% and we'll compile it for you on the right. If you give
% someone the link to this page, they can edit at the same
% time. See the help menu above for more info. Enjoy!
%
%%%%%%%%%%%%%%%%%%%%%%%%%%%%%%%%%%%%%%%%%%%%%%%%%%%%%%%%%%%%%%%

% --------------------------------------------------------------
% This is all preamble stuff that you don't have to worry about.
% Head down to where it says "Start here"
% --------------------------------------------------------------
 
\documentclass[12pt]{article}
 
\usepackage[margin=1in]{geometry}
\usepackage{amsmath,amsthm,amssymb}

\usepackage{listings}
\usepackage{xcolor}

\usepackage{tikz}
\usetikzlibrary{shapes,positioning}

\tikzset{ell/.style={circle,draw,minimum height=0.5cm,minimum width=0.5cm,inner sep=0.2cm}}

%New colors defined below
\definecolor{codegreen}{rgb}{0,0.6,0}
\definecolor{codegray}{rgb}{0.5,0.5,0.5}
\definecolor{codepurple}{rgb}{0.58,0,0.82}
\definecolor{backcolour}{rgb}{0.95,0.95,0.92}

%Code listing style named "mystyle"
\lstdefinestyle{mystyle}{
  backgroundcolor=\color{backcolour}, commentstyle=\color{codegreen},
  keywordstyle=\color{magenta},
  numberstyle=\tiny\color{codegray},
  stringstyle=\color{codepurple},
  basicstyle=\ttfamily\footnotesize,
  breakatwhitespace=false,         
  breaklines=true,                 
  captionpos=b,                    
  keepspaces=true,                 
  numbers=left,                    
  numbersep=5pt,                  
  showspaces=false,                
  showstringspaces=false,
  showtabs=false,                  
  tabsize=2
}

%"mystyle" code listing set
\lstset{style=mystyle}

 
\newcommand{\N}{\mathbb{N}}
\newcommand{\Z}{\mathbb{Z}}
 
\newenvironment{theorem}[2][Theorem]{\begin{trivlist}
\item[\hskip \labelsep {\bfseries #1}\hskip \labelsep {\bfseries #2.}]}{\end{trivlist}}
\newenvironment{lemma}[2][Lemma]{\begin{trivlist}
\item[\hskip \labelsep {\bfseries #1}\hskip \labelsep {\bfseries #2.}]}{\end{trivlist}}
\newenvironment{exercise}[2][Exercise]{\begin{trivlist}
\item[\hskip \labelsep {\bfseries #1}\hskip \labelsep {\bfseries #2.}]}{\end{trivlist}}
\newenvironment{problem}[2][Problem]{\begin{trivlist}
\item[\hskip \labelsep {\bfseries #1}\hskip \labelsep {\bfseries #2.}]}{\end{trivlist}}
\newenvironment{question}[2][Question]{\begin{trivlist}
\item[\hskip \labelsep {\bfseries #1}\hskip \labelsep {\bfseries #2.}]}{\end{trivlist}}
\newenvironment{corollary}[2][Corollary]{\begin{trivlist}
\item[\hskip \labelsep {\bfseries #1}\hskip \labelsep {\bfseries #2.}]}{\end{trivlist}}

\newenvironment{solution}{\begin{proof}[Solution]}{\end{proof}}
 
\begin{document}
 
% --------------------------------------------------------------
%                         Start here
% --------------------------------------------------------------
 
\title{Discussion 12}%replace X with the appropriate number
\author{Mengxiang Jiang\\ %replace with your name
CSEN 5303 Foundations of Computer Science} %if necessary, replace with your course title
 
\maketitle

\begin{problem}{statement}
    Feel free to answer one of the following questions:
\begin{quote}
1. Discuss the direct proof method. Give an illustrative example.\\
2. Discuss the indirect proof (or proof by contraposition) method. Give an illustrative
example.
\end{quote}
\end{problem}

\begin{problem}{2}
An indirect proof can take two forms, a proof by contraposition or a proof by contradiction.
A proof by contraposition rearranges the original implication into its contrapositive form.
Namely $P \rightarrow Q$ becomes $\neg Q \rightarrow \neg P$. 
Since the contrapositive is logically equivalent to the original, one can then proceed with a direct proof of the contrapositive and thereby prove the original.\\
\\Example: For any integer $n$, if $n^2$ is odd, then $n$ is odd\\
Contrapositive: For any integer $n$, if $n$ is even, then $n^2$ is even\\
Since $n$ is even, let $n = 2k$ for some integer $k$, then $n^2 = (2k)^2 = 4k^2 = 2(2k^2)$, so $n^2$ must be even. $\qed$
\\\\A proof by contradiction starts assuming the original premise implies the negation of the original conclusion. 
Then from the negated conclusion, derive an inconsistency with the original premise, thereby arriving at a contradiction and proving the original.
Namely if the original was $P \rightarrow Q$, then assume $P \rightarrow \neg Q$, and from $\neg Q$ derive $\neg P$.\\
\\Example: For any integer $n$, if $n^2$ is odd, then $n$ is odd\\
Assume for contradiction: For any integer $n$, if $n^2$ is odd, then $n$ is even.\\
Since $n$ is even, let $n = 2k$ for some integer $k$, then $n^2 = (2k)^2 = 4k^2 = 2(2k^2)$, so $n^2$ must be even.
But we assumed $n^2$ is odd, and $n^2$ cannot be both odd and even, therefore contradiction. $\qed$\\\\
\end{problem}
% --------------------------------------------------------------
%     You don't have to mess with anything below this line.
% --------------------------------------------------------------
 
\end{document}