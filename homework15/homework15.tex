%%%%%%%%%%%%%%%%%%%%%%%%%%%%%%%%%%%%%%%%%%%%%%%%%%%%%%%%%%%%%%%
%
% Welcome to writeLaTeX --- just edit your LaTeX on the left,
% and we'll compile it for you on the right. If you give
% someone the link to this page, they can edit at the same
% time. See the help menu above for more info. Enjoy!
%
%%%%%%%%%%%%%%%%%%%%%%%%%%%%%%%%%%%%%%%%%%%%%%%%%%%%%%%%%%%%%%%

% --------------------------------------------------------------
% This is all preamble stuff that you don't have to worry about.
% Head down to where it says "Start here"
% --------------------------------------------------------------
 
\documentclass[12pt]{article}
 
\usepackage[margin=1in]{geometry}
\usepackage{amsmath,amsthm,amssymb}

\usepackage{listings}
\usepackage{xcolor}

\usepackage{tikz}
\usetikzlibrary{shapes,positioning}

\tikzset{ell/.style={circle,draw,minimum height=0.5cm,minimum width=0.5cm,inner sep=0.2cm}}

%New colors defined below
\definecolor{codegreen}{rgb}{0,0.6,0}
\definecolor{codegray}{rgb}{0.5,0.5,0.5}
\definecolor{codepurple}{rgb}{0.58,0,0.82}
\definecolor{backcolour}{rgb}{0.95,0.95,0.92}

%Code listing style named "mystyle"
\lstdefinestyle{mystyle}{
  backgroundcolor=\color{backcolour}, commentstyle=\color{codegreen},
  keywordstyle=\color{magenta},
  numberstyle=\tiny\color{codegray},
  stringstyle=\color{codepurple},
  basicstyle=\ttfamily\footnotesize,
  breakatwhitespace=false,         
  breaklines=true,                 
  captionpos=b,                    
  keepspaces=true,                 
  numbers=left,                    
  numbersep=5pt,                  
  showspaces=false,                
  showstringspaces=false,
  showtabs=false,                  
  tabsize=2
}

%"mystyle" code listing set
\lstset{style=mystyle}

 
\newcommand{\N}{\mathbb{N}}
\newcommand{\Z}{\mathbb{Z}}
 
\newenvironment{theorem}[2][Theorem]{\begin{trivlist}
\item[\hskip \labelsep {\bfseries #1}\hskip \labelsep {\bfseries #2.}]}{\end{trivlist}}
\newenvironment{lemma}[2][Lemma]{\begin{trivlist}
\item[\hskip \labelsep {\bfseries #1}\hskip \labelsep {\bfseries #2.}]}{\end{trivlist}}
\newenvironment{exercise}[2][Exercise]{\begin{trivlist}
\item[\hskip \labelsep {\bfseries #1}\hskip \labelsep {\bfseries #2.}]}{\end{trivlist}}
\newenvironment{problem}[2][Problem]{\begin{trivlist}
\item[\hskip \labelsep {\bfseries #1}\hskip \labelsep {\bfseries #2.}]}{\end{trivlist}}
\newenvironment{question}[2][Question]{\begin{trivlist}
\item[\hskip \labelsep {\bfseries #1}\hskip \labelsep {\bfseries #2.}]}{\end{trivlist}}
\newenvironment{corollary}[2][Corollary]{\begin{trivlist}
\item[\hskip \labelsep {\bfseries #1}\hskip \labelsep {\bfseries #2.}]}{\end{trivlist}}

\newenvironment{solution}{\begin{proof}[Solution]}{\end{proof}}
 
\begin{document}
 
% --------------------------------------------------------------
%                         Start here
% --------------------------------------------------------------
 
\title{Homework 15}%replace X with the appropriate number
\author{Mengxiang Jiang\\ %replace with your name
CSEN 5303 Foundations of Computer Science} %if necessary, replace with your course title
 
\maketitle

\begin{problem}{1}
What is wrong with the following ``proof" by mathematical induction?
    \begin{quote}
    We will prove that all computers are built by the same manufacturer. In particular, we will prove
    that in any collection of $n$ computers, where $n$ is a positive integer, all of the computers are built
    by the same manufacturer. We first prove $P(1)$. This is a trivial process because in any collection
    consisting of one computer, there is only one manufacturer. Now, we assume $P(k)$, i.e., the
    inductive hypothesis. That is, in any collection of $k$ computers, all the computers are built by the
    same manufacturer. To prove $P(k + 1)$, we consider any collection of $k + 1$ computers. Pull one of
    these $k + 1$ computers (call it HAL) out of the collection. By our assumption, the remaining $k$
    computers all have the same manufacturer. Let HAL change places with one of these $k$ computers.
    In the new group of $k$ computers, all have the same manufacturer. Thus, HAL's manufacturer is
    the same one that produced all the other computers, and all $k + 1$ computers have the same
    manufacturer. 
    \end{quote}
The issue is that during the inductive step, proving $P(k+1)$ does not immediately follow from $P(k)$
when $k=1$. This is because when you have 2 computers, and you take 1 out, the remaining computer will trivially be from the same manufacturer as itself.
When you put the taken out computer back in and take out the one that wasn't taken, again, you have one computer from the same manufacturer as itself, but not necessarily the same as the one taken out.
\end{problem}
\pagebreak
\begin{problem}{2}
    An obscure tribe has only three words in its language, $moon$, $noon$, and $soon$. New
words are composed by juxtaposing these words in any order, as in $soonnoonmoonnoon$. Any such
juxtaposition is a legal word.

    \begin{enumerate}
    \item Use mathematical induction (also called \emph{first principle of induction}) 
    on the number of subwords in the word to prove that any word in this language has an even number of $o$'s.\\\\
    Basis step: $P(1)$: when there is only one word in the juxtaposition, 
    it has to be either $moon$, $noon$, or $soon$, 
    all of which have exactly 2 $o$'s, therefore even and $P(1)$ is true.\\
    Inductive hypothesis: Assume $P(k)$ is true, 
    namely any juxtaposition of $k$ words has an even number of $o$'s.
    Namely any juxtaposition $w_1w_2\ldots w_k$ has $2m$ $o$'s, for some integer $m$.\\
    Inductive step: Need to show $P(k+1)$ is true using $P(k)$. 
    For any juxtaposition of $k+1$ words, $w_1w_2 \ldots w_kw_{k+1}$, 
    the first $k$ words, $w_1w_2\ldots w_k$, has an even number, $2m$, of $o$'s ($P(k)$).
    Since $w_{k+1}$ is one word, it has exactly 2 $o$'s, so when appended to the end of $w_1w_2\ldots w_k$,
    we have a total of $2m+2 = 2(m+1)$ $o$'s, which is even. Therefore $P(k+1)$ is true. $\qed$
    
    \item Use strong induction (also called \emph{second principle of induction}) 
    on the number of subwords in the word to prove that any word in this language has an even number of $o$'s.\\\\
    Basis step: $P(1)$: when there is only one word in the juxtaposition, 
    it has to be either $moon$, $noon$, or $soon$, 
    all of which have exactly 2 $o$'s, therefore even and $P(1)$ is true.\\
    Inductive hypotheses: Assume $P(1),P(2),\ldots,P(k)$ is true, 
    namely any juxtaposition of 1 to $k$ words has an even number of $o$'s.\\
    Inductive step: Need to show $P(k+1)$ is true using $P(1),P(2),\ldots P(k)$.
    For any juxtaposition of $k+1$ words, $w_1w_2 \ldots w_kw_{k+1}$,
    we can break it into the concatenation of two smaller juxtapositions, namely $jux_a = w_1w_2\ldots w_i$
    and $jux_b = w_{i+1}\ldots w_kw_{k+1}$. 
    The number of words in $jux_a$ is less than or equal to $k$, and the same for $jux_b$.
    So the inductive hypotheses apply to both, so let's call $2l$ the number of $o$'s for $jux_a$ and $2m$ for $jux_b$.
    Therefore the original juxtaposition with $k+1$ words has $2l + 2m = 2lm$ $o$'s, an even number.
    Therefore $P(k+1)$ is true. $\qed$
    \end{enumerate}
\end{problem}
\pagebreak
\begin{problem}{3}
Your bank ATM delivers cash using only \$20 and \$50 bills.
Prove that you can collect, in addition to \$20, any multiple of \$10 that is \$40 or greater.\\\\
Basis step: $P(40)$: $40 = 20 + 20$, so \$40 is an amount that is collectable, so $P(40)$ is true.
$P(50)$: $50 = 50$, so \$50 is an amount that is also collectable, so $P(50)$ is also true.\\
Inductive hypotheses: Assume $P(40),P(50),\ldots P(10k)$ is true,
namely any multiple of \$10 greater than \$40 and less than \$$10k$ is collectable.\\
Inductive step: Need to show $P(10(k+1))$ is true using $P(40),P(50),\ldots P(10k)$.
Since \$10(k-1) is collectable ($P(10(k-1))$) and the difference between \$10(k+1) and \$10(k-1) is \$20,
we can simply add an additional \$20 to the \$10(k-1) solution for a solution to \$10(k+1), therefore $P(10(k+1))$ is true when $k>4$.
The corner case when $k=4$ is handled by the second basis step of $P(50)$, since \$30 is not collectable but \$50 is. $\qed$ 
\end{problem}

% --------------------------------------------------------------
%     You don't have to mess with anything below this line.
% --------------------------------------------------------------
 
\end{document}