%%%%%%%%%%%%%%%%%%%%%%%%%%%%%%%%%%%%%%%%%%%%%%%%%%%%%%%%%%%%%%%
%
% Welcome to writeLaTeX --- just edit your LaTeX on the left,
% and we'll compile it for you on the right. If you give
% someone the link to this page, they can edit at the same
% time. See the help menu above for more info. Enjoy!
%
%%%%%%%%%%%%%%%%%%%%%%%%%%%%%%%%%%%%%%%%%%%%%%%%%%%%%%%%%%%%%%%

% --------------------------------------------------------------
% This is all preamble stuff that you don't have to worry about.
% Head down to where it says "Start here"
% --------------------------------------------------------------
 
\documentclass[12pt]{article}
 
\usepackage[margin=1in]{geometry}
\usepackage{amsmath,amsthm,amssymb}

\usepackage{listings}
\usepackage{xcolor}

\usepackage{tikz}
\usetikzlibrary{shapes,positioning}

\tikzset{ell/.style={circle,draw,minimum height=0.5cm,minimum width=0.5cm,inner sep=0.2cm}}

%New colors defined below
\definecolor{codegreen}{rgb}{0,0.6,0}
\definecolor{codegray}{rgb}{0.5,0.5,0.5}
\definecolor{codepurple}{rgb}{0.58,0,0.82}
\definecolor{backcolour}{rgb}{0.95,0.95,0.92}

%Code listing style named "mystyle"
\lstdefinestyle{mystyle}{
  backgroundcolor=\color{backcolour}, commentstyle=\color{codegreen},
  keywordstyle=\color{magenta},
  numberstyle=\tiny\color{codegray},
  stringstyle=\color{codepurple},
  basicstyle=\ttfamily\footnotesize,
  breakatwhitespace=false,         
  breaklines=true,                 
  captionpos=b,                    
  keepspaces=true,                 
  numbers=left,                    
  numbersep=5pt,                  
  showspaces=false,                
  showstringspaces=false,
  showtabs=false,                  
  tabsize=2
}

%"mystyle" code listing set
\lstset{style=mystyle}

 
\newcommand{\N}{\mathbb{N}}
\newcommand{\Z}{\mathbb{Z}}
 
\newenvironment{theorem}[2][Theorem]{\begin{trivlist}
\item[\hskip \labelsep {\bfseries #1}\hskip \labelsep {\bfseries #2.}]}{\end{trivlist}}
\newenvironment{lemma}[2][Lemma]{\begin{trivlist}
\item[\hskip \labelsep {\bfseries #1}\hskip \labelsep {\bfseries #2.}]}{\end{trivlist}}
\newenvironment{exercise}[2][Exercise]{\begin{trivlist}
\item[\hskip \labelsep {\bfseries #1}\hskip \labelsep {\bfseries #2.}]}{\end{trivlist}}
\newenvironment{problem}[2][Problem]{\begin{trivlist}
\item[\hskip \labelsep {\bfseries #1}\hskip \labelsep {\bfseries #2.}]}{\end{trivlist}}
\newenvironment{question}[2][Question]{\begin{trivlist}
\item[\hskip \labelsep {\bfseries #1}\hskip \labelsep {\bfseries #2.}]}{\end{trivlist}}
\newenvironment{corollary}[2][Corollary]{\begin{trivlist}
\item[\hskip \labelsep {\bfseries #1}\hskip \labelsep {\bfseries #2.}]}{\end{trivlist}}

\newenvironment{solution}{\begin{proof}[Solution]}{\end{proof}}
 
\begin{document}
 
% --------------------------------------------------------------
%                         Start here
% --------------------------------------------------------------
 
\title{Discussion 8}%replace X with the appropriate number
\author{Mengxiang Jiang\\ %replace with your name
CSEN 5303 Foundations of Computer Science} %if necessary, replace with your course title
 
\maketitle

\begin{problem}{statement}
    Consider the statement:\\
    ``There is a person x who is a student in CSEN 5303 and has visited Mexico".\\
    Explain why the answer cannot be $\exists x (S(x) \rightarrow M(x))$.\\\\
$S(x): x$ is a student in CSEN 5303.\\
$M(x): x$ has visited Mexico.\\
``There is a person x who is a student in CSEN 5303 and has visited Mexico": $\exists x (S(x) \land M(x))$
\begin{center}
    \begin{tabular}{|c c c c|} 
     \hline
    $S(x)$ & $M(x)$ & $S(x) \land M(x)$ & $S(x) \rightarrow M(x)$ \\
     \hline
     T & T & T & T \\ 
     T & F & F & F\\
     F & T & F & T\\
     F & F & F & T\\
     \hline
    \end{tabular}
\end{center}
Since the truth table values are different, the two statements are not equivalent.\\\\
Note: $\exists x (S(x) \rightarrow M(x))$ is known as the Drinker's Paradox,
and is true in two cases. Case one: there is a CSEN 5303 student who visited Mexico, which intuitively makes sense.
Case two: there is a person who is not a CSEN 5303 student (regardless of whether this person visited Mexico or not).
This second case is highly unintuitive gives the paradox its name.

\end{problem}
% --------------------------------------------------------------
%     You don't have to mess with anything below this line.
% --------------------------------------------------------------
 
\end{document}