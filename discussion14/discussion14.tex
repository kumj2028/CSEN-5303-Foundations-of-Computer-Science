%%%%%%%%%%%%%%%%%%%%%%%%%%%%%%%%%%%%%%%%%%%%%%%%%%%%%%%%%%%%%%%
%
% Welcome to writeLaTeX --- just edit your LaTeX on the left,
% and we'll compile it for you on the right. If you give
% someone the link to this page, they can edit at the same
% time. See the help menu above for more info. Enjoy!
%
%%%%%%%%%%%%%%%%%%%%%%%%%%%%%%%%%%%%%%%%%%%%%%%%%%%%%%%%%%%%%%%

% --------------------------------------------------------------
% This is all preamble stuff that you don't have to worry about.
% Head down to where it says "Start here"
% --------------------------------------------------------------
 
\documentclass[12pt]{article}
 
\usepackage[margin=1in]{geometry}
\usepackage{amsmath,amsthm,amssymb}

\usepackage{listings}
\usepackage{xcolor}

\usepackage{tikz}
\usetikzlibrary{shapes,positioning}

\tikzset{ell/.style={circle,draw,minimum height=0.5cm,minimum width=0.5cm,inner sep=0.2cm}}

%New colors defined below
\definecolor{codegreen}{rgb}{0,0.6,0}
\definecolor{codegray}{rgb}{0.5,0.5,0.5}
\definecolor{codepurple}{rgb}{0.58,0,0.82}
\definecolor{backcolour}{rgb}{0.95,0.95,0.92}

%Code listing style named "mystyle"
\lstdefinestyle{mystyle}{
  backgroundcolor=\color{backcolour}, commentstyle=\color{codegreen},
  keywordstyle=\color{magenta},
  numberstyle=\tiny\color{codegray},
  stringstyle=\color{codepurple},
  basicstyle=\ttfamily\footnotesize,
  breakatwhitespace=false,         
  breaklines=true,                 
  captionpos=b,                    
  keepspaces=true,                 
  numbers=left,                    
  numbersep=5pt,                  
  showspaces=false,                
  showstringspaces=false,
  showtabs=false,                  
  tabsize=2
}

%"mystyle" code listing set
\lstset{style=mystyle}

 
\newcommand{\N}{\mathbb{N}}
\newcommand{\Z}{\mathbb{Z}}
 
\newenvironment{theorem}[2][Theorem]{\begin{trivlist}
\item[\hskip \labelsep {\bfseries #1}\hskip \labelsep {\bfseries #2.}]}{\end{trivlist}}
\newenvironment{lemma}[2][Lemma]{\begin{trivlist}
\item[\hskip \labelsep {\bfseries #1}\hskip \labelsep {\bfseries #2.}]}{\end{trivlist}}
\newenvironment{exercise}[2][Exercise]{\begin{trivlist}
\item[\hskip \labelsep {\bfseries #1}\hskip \labelsep {\bfseries #2.}]}{\end{trivlist}}
\newenvironment{problem}[2][Problem]{\begin{trivlist}
\item[\hskip \labelsep {\bfseries #1}\hskip \labelsep {\bfseries #2.}]}{\end{trivlist}}
\newenvironment{question}[2][Question]{\begin{trivlist}
\item[\hskip \labelsep {\bfseries #1}\hskip \labelsep {\bfseries #2.}]}{\end{trivlist}}
\newenvironment{corollary}[2][Corollary]{\begin{trivlist}
\item[\hskip \labelsep {\bfseries #1}\hskip \labelsep {\bfseries #2.}]}{\end{trivlist}}

\newenvironment{solution}{\begin{proof}[Solution]}{\end{proof}}
 
\begin{document}
 
% --------------------------------------------------------------
%                         Start here
% --------------------------------------------------------------
 
\title{Discussion 14}%replace X with the appropriate number
\author{Mengxiang Jiang\\ %replace with your name
CSEN 5303 Foundations of Computer Science} %if necessary, replace with your course title
 
\maketitle

\begin{problem}{statement}
    Discuss the mathematical induction proof method. Provide an illustrative example.
    \begin{quote}
        Mathematical induction is a proof technique that applies to statements about integers. 
        Suppose $P(n)$ is a statement about the integer $n$, 
        and we want to prove it is true for all $n$ in a certain iteratable domain 
        (usually this is the positive integers or can be converted to such).
        The following are the steps of the proof:\\\\
        1. Basis step: prove $P(1)$ is true.\\
        2. Inductive hypothesis: assume $P(k)$ is true for any $k \geq 1$\\
        3. Inductive step: Use $P(k)$ to prove $P(k+1)$. Once this is shown, the proof is complete.\\\\
        The proof takes advantage of the fact that any positive integer is by definition either 1 
        or a successor of another positive integer, thereby covering all possible cases.\\\\
        An illustrative example of proof by induction is the Fibonacci inequality:
        Let the \emph{Fibonacci sequence} be defined as $F_0 = 0$, $F_1 = 1$ and all subsequent terms be $F_{n+2} = F_n + F_{n+1}$.\\
        Let $\phi$ be $\frac{1+\sqrt{5}}{2}$.\\
        Then $F_n \leq \phi^{n-1}$ for all positive integers $n$.\\\\
        1. Basis step: $P(1)$ is the claim $F_1 = 1 \leq \phi^{1-1} = \phi^0 = 1$, so $P(1)$ is true.\\
        However there is a second basis step required, namely $P(2)$:\\
        $F_2 = F_0 + F_1 = 0 + 1 = 1 \leq \phi^{2-1} = \phi^1 = \frac{1+\sqrt{5}}{2}$.\\
        This is true since $\sqrt{5} > \sqrt{4} = 2 \rightarrow \frac{1+\sqrt{5}}{2} > \frac{1+2}{2} > \frac{2}{2} = 1$.\\\\
        2. Inductive hypothesis: we'll need to assume two inductive hypotheses, 
        namely $P(k)$ and $P(k+1)$ is true, which is why we needed two base cases for $k \geq 1$ and $(k+1) \geq 2$.\\
        \pagebreak
        \\3. Inductive step: $P(k+2)$: $F_{k+2} \leq \phi^{k+2-1}$. To show this we have:\\
        $F_{k+2} = F_k + F_{k+1} \leq \phi^{k-1} + \phi^k$ from the inductive hypotheses.\\
        One additional fact we have to make use of is the fact that $\phi$ is known as the golden ratio with the following property:\\
        $1 + \phi = \phi^2$\\
        We see that we can factor $\phi^{k-1} + \phi^k = \phi^{k-1}(1+\phi)$\\
        Substituting $1 + \phi = \phi^2$ in, we have $\phi^{k-1}(\phi^2) = \phi^{k-1+2} = \phi^{k+2-1}$ as desired.\\
        $\therefore P(k+2) \equiv F_{k+2} \leq \phi^{k+2-1}$ is true $\qed$

    \end{quote}
\end{problem}
% --------------------------------------------------------------
%     You don't have to mess with anything below this line.
% --------------------------------------------------------------
 
\end{document}