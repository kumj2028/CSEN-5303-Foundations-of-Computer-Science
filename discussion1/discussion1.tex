%%%%%%%%%%%%%%%%%%%%%%%%%%%%%%%%%%%%%%%%%%%%%%%%%%%%%%%%%%%%%%%
%
% Welcome to writeLaTeX --- just edit your LaTeX on the left,
% and we'll compile it for you on the right. If you give
% someone the link to this page, they can edit at the same
% time. See the help menu above for more info. Enjoy!
%
%%%%%%%%%%%%%%%%%%%%%%%%%%%%%%%%%%%%%%%%%%%%%%%%%%%%%%%%%%%%%%%

% --------------------------------------------------------------
% This is all preamble stuff that you don't have to worry about.
% Head down to where it says "Start here"
% --------------------------------------------------------------
 
\documentclass[12pt]{article}
 
\usepackage[margin=1in]{geometry}
\usepackage{amsmath,amsthm,amssymb}
 
\newcommand{\N}{\mathbb{N}}
\newcommand{\Z}{\mathbb{Z}}
 
\newenvironment{theorem}[2][Theorem]{\begin{trivlist}
\item[\hskip \labelsep {\bfseries #1}\hskip \labelsep {\bfseries #2.}]}{\end{trivlist}}
\newenvironment{lemma}[2][Lemma]{\begin{trivlist}
\item[\hskip \labelsep {\bfseries #1}\hskip \labelsep {\bfseries #2.}]}{\end{trivlist}}
\newenvironment{exercise}[2][Exercise]{\begin{trivlist}
\item[\hskip \labelsep {\bfseries #1}\hskip \labelsep {\bfseries #2.}]}{\end{trivlist}}
\newenvironment{problem}[2][Problem]{\begin{trivlist}
\item[\hskip \labelsep {\bfseries #1}\hskip \labelsep {\bfseries #2.}]}{\end{trivlist}}
\newenvironment{question}[2][Question]{\begin{trivlist}
\item[\hskip \labelsep {\bfseries #1}\hskip \labelsep {\bfseries #2.}]}{\end{trivlist}}
\newenvironment{corollary}[2][Corollary]{\begin{trivlist}
\item[\hskip \labelsep {\bfseries #1}\hskip \labelsep {\bfseries #2.}]}{\end{trivlist}}

\newenvironment{solution}{\begin{proof}[Solution]}{\end{proof}}
 
\begin{document}
 
% --------------------------------------------------------------
%                         Start here
% --------------------------------------------------------------
 
\title{Discussion 1}%replace X with the appropriate number
\author{Mengxiang Jiang\\ %replace with your name
CSEN 5303 Foundations of Computer Science} %if necessary, replace with your course title
 
\maketitle
 
\begin{problem}{statement} %You can use theorem, exercise, problem, or question here.  Modify x.yz to be whatever number you are proving
Feel free to answer one of the following questions:

1. Define the concept of algorithm. Give an illustrative example.

2. Define the concept of program. What is the difference between algorithm and program?
\end{problem}
 
 \begin{problem}{1}
 In \emph{The Art of Computer Programming} by Donald Knuth, he defines an algorithm as a ``set of rules that gives a sequence of operations for solving a specific type of problem" with ``five important features":\\
 
 The first feature is \emph{finiteness}, where the algorithm must terminate in a finite number of operations. \\
 
 The second feature is \emph{definiteness}, where each operation must be precisely and unambiguously defined. \\
 
 The third feature is \emph{input}, where the algorithm can take in zero or more quantities of information to process. \\
 
 The fourth feature is \emph{output}, where the algorithm has to produce one or more quantities of information as the result.\\
 
 The fifth and final feature is \emph{effectiveness}, which is stated in the book as ``operations must all be sufficiently basic that they can in principle be done exactly and in a finite length of time by someone using pencil and paper".\\\\
 The illustrative example given in the book is Euclid's greatest common divisor algorithm:\\\\
 \textbf{Algorithm E} (\emph{Euclid's algorithm}). Given two positive integers $m$ and $n$, find
their \emph{greatest common divisor}, that is, the largest positive integer that evenly
divides both $m$ and $n$.\\
\textbf{E1.} [Find remainder.] Divide $m$ by $n$ and let $r$ be the remainder. (We will have
$0 \leq r < n$.)\\
\textbf{E2.}  [Is it zero?] If $r$ = 0, the algorithm terminates; $n$ is the answer.\\
\textbf{E3.} [Reduce.] Set $m \leftarrow n$, $n \leftarrow r$, and go back to step El.
\\\\\\
This algorithm meets all 5 of the features:\\\\
1. Since $m$ and $n$ are strictly decreasing after every iteration, and since a decreasing sequence of positive integers must terminate in finite time, the algorithm is finite.\\
2. Although the algorithm is given in English (with all it's ambiguities), the choice of words like \emph{positive}, \emph{integer}, and \emph{remainder} are fairly specific in mathematical meaning to avoid misunderstanding.\\
3. The inputs for the algorithm are $m$ and $n$.\\
4. The output of the algorithm is $n$, once the algorithm terminates.\\
5. Since positive integers can be represented on paper in a finite form and can be divided using long division in finite time, a person in theory can perform all the steps of the algorithm from start to completion.
\\\\
Example execution of Algorithm E with input $m = 119$ and $n = 544$
\begin{center}
\begin{tabular}{|c c c c|} 
 \hline
 Iteration & $m$ & $n$ & $r$ \\
 \hline
 1 & 119 & 544 & 119 \\ 
 2 & 544 & 119 & 68 \\
 3 & 119 & 68 & 51 \\
 4 & 68 & 51 & 17 \\
 5 & 51 & 17 & 0 \\
 \hline
\end{tabular}
\end{center}
 
 \end{problem}
 
% --------------------------------------------------------------
%     You don't have to mess with anything below this line.
% --------------------------------------------------------------
 
\end{document}