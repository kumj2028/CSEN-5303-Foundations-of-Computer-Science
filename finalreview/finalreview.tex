%%%%%%%%%%%%%%%%%%%%%%%%%%%%%%%%%%%%%%%%%%%%%%%%%%%%%%%%%%%%%%%
%
% Welcome to writeLaTeX --- just edit your LaTeX on the left,
% and we'll compile it for you on the right. If you give
% someone the link to this page, they can edit at the same
% time. See the help menu above for more info. Enjoy!
%
%%%%%%%%%%%%%%%%%%%%%%%%%%%%%%%%%%%%%%%%%%%%%%%%%%%%%%%%%%%%%%%

% --------------------------------------------------------------
% This is all preamble stuff that you don't have to worry about.
% Head down to where it says "Start here"
% --------------------------------------------------------------
 
\documentclass[12pt]{article}
 
\usepackage[margin=1in]{geometry}
\usepackage{amsmath,amsthm,amssymb}

\usepackage{listings}
\usepackage{xcolor}

\usepackage{tikz}
\usetikzlibrary{shapes,positioning}

\tikzset{ell/.style={circle,draw,minimum height=0.5cm,minimum width=0.5cm,inner sep=0.2cm}}

%New colors defined below
\definecolor{codegreen}{rgb}{0,0.6,0}
\definecolor{codegray}{rgb}{0.5,0.5,0.5}
\definecolor{codepurple}{rgb}{0.58,0,0.82}
\definecolor{backcolour}{rgb}{0.95,0.95,0.92}

%Code listing style named "mystyle"
\lstdefinestyle{mystyle}{
  backgroundcolor=\color{backcolour}, commentstyle=\color{codegreen},
  keywordstyle=\color{magenta},
  numberstyle=\tiny\color{codegray},
  stringstyle=\color{codepurple},
  basicstyle=\ttfamily\footnotesize,
  breakatwhitespace=false,         
  breaklines=true,                 
  captionpos=b,                    
  keepspaces=true,                 
  numbers=left,                    
  numbersep=5pt,                  
  showspaces=false,                
  showstringspaces=false,
  showtabs=false,                  
  tabsize=2
}

%"mystyle" code listing set
\lstset{style=mystyle}

 
\newcommand{\N}{\mathbb{N}}
\newcommand{\Z}{\mathbb{Z}}
 
\newenvironment{theorem}[2][Theorem]{\begin{trivlist}
\item[\hskip \labelsep {\bfseries #1}\hskip \labelsep {\bfseries #2.}]}{\end{trivlist}}
\newenvironment{lemma}[2][Lemma]{\begin{trivlist}
\item[\hskip \labelsep {\bfseries #1}\hskip \labelsep {\bfseries #2.}]}{\end{trivlist}}
\newenvironment{exercise}[2][Exercise]{\begin{trivlist}
\item[\hskip \labelsep {\bfseries #1}\hskip \labelsep {\bfseries #2.}]}{\end{trivlist}}
\newenvironment{problem}[2][Problem]{\begin{trivlist}
\item[\hskip \labelsep {\bfseries #1}\hskip \labelsep {\bfseries #2.}]}{\end{trivlist}}
\newenvironment{question}[2][Question]{\begin{trivlist}
\item[\hskip \labelsep {\bfseries #1}\hskip \labelsep {\bfseries #2.}]}{\end{trivlist}}
\newenvironment{corollary}[2][Corollary]{\begin{trivlist}
\item[\hskip \labelsep {\bfseries #1}\hskip \labelsep {\bfseries #2.}]}{\end{trivlist}}

\newenvironment{solution}{\begin{proof}[Solution]}{\end{proof}}
 
\begin{document}
 
% --------------------------------------------------------------
%                         Start here
% --------------------------------------------------------------
 
\title{Final Review}%replace X with the appropriate number
\author{Mengxiang Jiang\\ %replace with your name
CSEN 5303 Foundations of Computer Science} %if necessary, replace with your course title
 
\maketitle

\begin{problem}{6}
    Let $P(n)$ be the statement:
    $$\log{n!} > \frac{n\log{n}}{4},\text{ for $n > 4$}$$
    \begin{enumerate}
        \item What is the statement $P(5)$?\\
        $$P(5):\; \log{5!} > \frac{5\log{5}}{4}$$
        \item Show that $P(5)$ is true, completing the basis step of the proof.\\
        $$\log{5!} = \log(5\times4\times3\times2\times1) = \log(5\times24) > \log(5\times5) = \log5^2 = 2\log5 > \frac{5\log5}{4}$$
        \item What is the inductive hypothesis?\\
        $$\text{Assume $P(k)$ is true: }\log k! > \frac{k\log k}{4}$$
        \item What do you need to prove in the inductive step?\\
        $$\text{Need to prove $P(k+1)$ is true: }\log(k+1)! > \frac{(k+1)\log(k+1)}{4}$$
        \item Complete the inductive step. You must justify any single step in your proof. Otherwise,
        your answer is wrong. Show your work step by step.\\
        $$\log(k+1)! = \log((k+1)k!) = \log(k+1) + \log k! > \log(k+1) + \frac{k\log k}{4}$$
        $$\log(k+1) + \frac{k\log k}{4} = \frac{4\log(k+1) + k\log k}{4} = \frac{\log((k+1)^4k^k)}{4}$$
        \begin{lemma}{Q}
            Let $Q(n)$ be the statement:\\
            $$n^n > (n+1)^{n-1},\text{ for $n > 2$}$$
            Basis: $Q(3):\; 3^3 = 27 > (3+1)^{3-1} = 4^2 = 16$, so $Q(3)$ is true.\\
            Inductive: Assume $Q(k)$ is true: $k^k > (k+1)^{k-1}$\\
            Need to prove $Q(k+1)$: $(k+1)^{k+1} > (k+2)^k$
            $$(k+1)^{k+1} = k^k\left(1+\frac{1}{k}\right)^k(k+1) > (k+1)^{k-1}(k+1)\left(1+\frac{1}{k}\right)^k$$
            $$ = (k+1)^{k}\left(1+\frac{1}{k}\right)^k = \left(k+\frac{1}{k} + 2\right)^k > (k+2)^k \rightarrow \text{Q(k+1) is true.}$$
            $$[Q(3) \land (Q(k) \rightarrow Q(k+1))] \rightarrow \forall n Q(n)$$ 
        \end{lemma}
        Using \textbf{Lemma Q}, we can proceed as follows:\\
        $$\frac{\log((k+1)^4k^k)}{4} > \frac{\log((k+1)^4(k+1)^{k-1})}{4} = \frac{(k+3)\log(k+1)}{4} > \frac{(k+1)\log(k+1)}{4}$$
        Therefore $P(k+1)$ is true.\\
        $$[P(5) \land (P(k) \rightarrow P(k+1))] \rightarrow \forall n P(n)$$
    \end{enumerate}
\end{problem}

% --------------------------------------------------------------
%     You don't have to mess with anything below this line.
% --------------------------------------------------------------
 
\end{document}