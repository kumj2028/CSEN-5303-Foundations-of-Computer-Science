%%%%%%%%%%%%%%%%%%%%%%%%%%%%%%%%%%%%%%%%%%%%%%%%%%%%%%%%%%%%%%%
%
% Welcome to writeLaTeX --- just edit your LaTeX on the left,
% and we'll compile it for you on the right. If you give
% someone the link to this page, they can edit at the same
% time. See the help menu above for more info. Enjoy!
%
%%%%%%%%%%%%%%%%%%%%%%%%%%%%%%%%%%%%%%%%%%%%%%%%%%%%%%%%%%%%%%%

% --------------------------------------------------------------
% This is all preamble stuff that you don't have to worry about.
% Head down to where it says "Start here"
% --------------------------------------------------------------
 
\documentclass[12pt]{article}
 
\usepackage[margin=1in]{geometry}
\usepackage{amsmath,amsthm,amssymb}

\usepackage{listings}
\usepackage{xcolor}

\usepackage{tikz}
\usetikzlibrary{shapes,positioning}

\tikzset{ell/.style={circle,draw,minimum height=0.5cm,minimum width=0.5cm,inner sep=0.2cm}}
\tikzset{rec/.style={rectangle,draw,minimum height=0.5cm,minimum width=0.5cm,inner sep=0.2cm}}

%New colors defined below
\definecolor{codegreen}{rgb}{0,0.6,0}
\definecolor{codegray}{rgb}{0.5,0.5,0.5}
\definecolor{codepurple}{rgb}{0.58,0,0.82}
\definecolor{backcolour}{rgb}{0.95,0.95,0.92}

%Code listing style named "mystyle"
\lstdefinestyle{mystyle}{
  backgroundcolor=\color{backcolour}, commentstyle=\color{codegreen},
  keywordstyle=\color{magenta},
  numberstyle=\tiny\color{codegray},
  stringstyle=\color{codepurple},
  basicstyle=\ttfamily\footnotesize,
  breakatwhitespace=false,         
  breaklines=true,                 
  captionpos=b,                    
  keepspaces=true,                 
  numbers=left,                    
  numbersep=5pt,                  
  showspaces=false,                
  showstringspaces=false,
  showtabs=false,                  
  tabsize=2
}

%"mystyle" code listing set
\lstset{style=mystyle}

 
\newcommand{\N}{\mathbb{N}}
\newcommand{\Z}{\mathbb{Z}}
 
\newenvironment{theorem}[2][Theorem]{\begin{trivlist}
\item[\hskip \labelsep {\bfseries #1}\hskip \labelsep {\bfseries #2.}]}{\end{trivlist}}
\newenvironment{lemma}[2][Lemma]{\begin{trivlist}
\item[\hskip \labelsep {\bfseries #1}\hskip \labelsep {\bfseries #2.}]}{\end{trivlist}}
\newenvironment{exercise}[2][Exercise]{\begin{trivlist}
\item[\hskip \labelsep {\bfseries #1}\hskip \labelsep {\bfseries #2.}]}{\end{trivlist}}
\newenvironment{problem}[2][Problem]{\begin{trivlist}
\item[\hskip \labelsep {\bfseries #1}\hskip \labelsep {\bfseries #2.}]}{\end{trivlist}}
\newenvironment{question}[2][Question]{\begin{trivlist}
\item[\hskip \labelsep {\bfseries #1}\hskip \labelsep {\bfseries #2.}]}{\end{trivlist}}
\newenvironment{corollary}[2][Corollary]{\begin{trivlist}
\item[\hskip \labelsep {\bfseries #1}\hskip \labelsep {\bfseries #2.}]}{\end{trivlist}}

\newenvironment{solution}{\begin{proof}[Solution]}{\end{proof}}
 
\begin{document}
 
% --------------------------------------------------------------
%                         Start here
% --------------------------------------------------------------
 
\title{Homework 11}%replace X with the appropriate number
\author{Mengxiang Jiang\\ %replace with your name
CSEN 5303 Foundations of Computer Science} %if necessary, replace with your course title
 
\maketitle

\begin{problem}{1}
    Use one of the proof methods to prove the following results. 
\begin{quote}
    1. Prove that $n^2 + 1 \geq 2n$, where $n$ is a positive integer with $1 \leq n \leq 4$.
\end{quote}
Exhaustive proof: since $n$ is an integer between 1 and 4, inclusive, there are only 4 different cases to consider.\\
$n=1 \rightarrow n^2 + 1 = 1^2 + 1 = 2 \geq 2(1) = 2$\\
$n=2 \rightarrow n^2 + 1 = 2^2 + 1 = 5 \geq 2(2) = 4$\\
$n=3 \rightarrow n^2 + 1 = 3^2 + 1 = 10 \geq 2(3) = 6$\\
$n=4 \rightarrow n^2 + 1 = 4^2 + 1 = 17 \geq 2(4) = 8 \qed$
\begin{quote}
    2. Prove that if $x$ and $y$ are real numbers, then $max(x, y) + min(x, y) = x + y$.
\end{quote}
Proof by cases: since it is either the case that $x \geq y$ or $x < y$, we have two cases to consider.\\
$x \geq y \rightarrow max(x, y) = x$ and $min(x, y) = y \rightarrow max(x,y)+min(x,y) = x + y$\\
$x < y \rightarrow max(x,y) = y$ and $min(x, y) = x \rightarrow max(x,y) + min(x,y) = y + x = x + y\qed$\\
\end{problem}

\begin{problem}{2}
Consider the set of integers. 
\begin{quote}
    1. Use a direct proof to show that the sum of two even integers is even
\end{quote}
Let $a$ and $b$ be two even integers.\\
Since $a$ is even, $a = 2m$ for some integer $m$.\\
Likewise $b = 2n$ for some integer $n$.\\
$a + b = 2m + 2n = 2(m+n) \qed$
\begin{quote}
    2. Prove that if $m + n$ and $n + p$ are even integers, where $m$, $n$, and $p$ are integers, then $m + p$
is even. What kind if proof did you use?
\end{quote}
Direct proof:
Since $m + n$ and $n + p$ are even integers, their sum must be even from the previous proof:\\
$(m + n) + (n + p) = m + 2n + p = 2k$ for some integer $k$\\
Subtract $2n$ from both sides we get $m + p = 2k - 2n = 2(k-n)\qed$
\pagebreak
\begin{quote}
    3. Use a direct proof to show that every odd integer is the difference of two squares
\end{quote}
Every odd integer can be represented as $2k + 1$ for some integer $k$.\\
$(k + 1)^2 - k^2 = k^2 + 2k + 1 - k^2 = 2k + 1 \qed$

\begin{quote}
    4. Prove that if $n$ is an integer and $3n + 2$ is even, then $n$ is even using\\
    a) A proof by contraposition.
\end{quote}
$P$: $3n + 2$ is even.\\
$Q$: $n$ is even.\\
$\neg Q:$ $n$ is odd, so $n = 2k + 1$ for some integer $k$.\\
$3n + 2 = 3(2k + 1) + 2 = 6k + 3 + 2 = 6k + 4 + 1 = 2(3k + 2) + 1 \equiv \neg P$ \\
$\neg Q \rightarrow \neg P \qed$
\begin{quote}
    b) A proof by contradiction.
\end{quote}
$P$: $3n + 2$ is even.\\
$Q$: $n$ is even.\\
$P \land \neg Q:$ $3n + 2$ is even and $n$ is odd.\\
Since $n$ is odd, $n = 2k+1$ for some integer $k$\\
$3n + 2 = 3(2k + 1) + 2 = 6k + 3 + 2 = 6k + 4 + 1 = 2(3k + 2) + 1 \equiv \neg P$.\\
$P \land \neg P$ contradiction $\qed$.

\begin{quote}
    5. Prove that if $n$ is an integer, the following four statements are equivalent:\\
a) $n$ is even.\\
b) $n + 1$ is odd.\\
c) $3n + 1$ is odd.\\
d) $3n$ is even.
\end{quote}
So we need to show $a \leftrightarrow b \leftrightarrow c \leftrightarrow d$.\\
$a \rightarrow b$: Direct: If $n$ is even, $n = 2k$ for some integer $k$, so $n+1 = 2k+1$.\\
$b \rightarrow a$: Direct: If $n+1$ is odd, $n+1 = 2k + 1$ for some integer $k$, so $n = 2k$.\\
$a \rightarrow c$: Direct: If $n$ is even, $n = 2k$ for some integer $k$, so $3n + 1 = 3(2k) + 1 = 2(3k) + 1$.\\
$c \rightarrow a$: Contraposition: If $n$ is odd, $n = 2k + 1$ for some integer $k$, so $3n+1 = 3(2k+1) + 1 = 6k+3+1 = 6k+4 = 2(3k + 2)$.\\
$a \rightarrow d$: Direct: If $n$ is even, $n = 2k$ for some integer $k$, so $3n = 3(2k) = 2(3k)$.\\
$d \rightarrow a$: Contraposition: If $n$ is odd, $n = 2k + 1$ for some integer $k$, so $3n = 3(2k+1) = 6k + 3 = 6k + 2 + 1 = 2(3k + 1) + 1$.\\
The rest of missing implications can be found using the hypothetical syllogism inference rule (i.e. $(b \rightarrow a) \land (a \rightarrow c) \equiv b \rightarrow c$).
\pagebreak
\begin{quote}
    6. Find a counterexample to the statement “Every positive integer can be written as the sum
of the squares of three integers”.
\end{quote}
7 cannot be written as a sum of three squares since:\\
$0 + 0 + 0 = 0$\\
$0 + 0 + 1 = 1$\\
$0 + 1 + 1 = 2$\\
$1 + 1 + 1 = 3$\\
$0 + 0 + 4 = 4$\\
$0 + 1 + 4 = 5$\\
$1 + 1 + 4 = 6$\\
$0 + 4 + 4 = 8$\\
$1 + 4 + 4 = 9$\\
$4 + 4 + 4 = 12$\\
Since 0, 1, and 4 are the only squares less than 7, and all their combinations fail
\end{problem}
% --------------------------------------------------------------
%     You don't have to mess with anything below this line.
% --------------------------------------------------------------
 
\end{document}