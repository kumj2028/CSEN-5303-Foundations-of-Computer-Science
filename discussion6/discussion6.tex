%%%%%%%%%%%%%%%%%%%%%%%%%%%%%%%%%%%%%%%%%%%%%%%%%%%%%%%%%%%%%%%
%
% Welcome to writeLaTeX --- just edit your LaTeX on the left,
% and we'll compile it for you on the right. If you give
% someone the link to this page, they can edit at the same
% time. See the help menu above for more info. Enjoy!
%
%%%%%%%%%%%%%%%%%%%%%%%%%%%%%%%%%%%%%%%%%%%%%%%%%%%%%%%%%%%%%%%

% --------------------------------------------------------------
% This is all preamble stuff that you don't have to worry about.
% Head down to where it says "Start here"
% --------------------------------------------------------------
 
\documentclass[12pt]{article}
 
\usepackage[margin=1in]{geometry}
\usepackage{amsmath,amsthm,amssymb}

\usepackage{listings}
\usepackage{xcolor}

%New colors defined below
\definecolor{codegreen}{rgb}{0,0.6,0}
\definecolor{codegray}{rgb}{0.5,0.5,0.5}
\definecolor{codepurple}{rgb}{0.58,0,0.82}
\definecolor{backcolour}{rgb}{0.95,0.95,0.92}

%Code listing style named "mystyle"
\lstdefinestyle{mystyle}{
  backgroundcolor=\color{backcolour}, commentstyle=\color{codegreen},
  keywordstyle=\color{magenta},
  numberstyle=\tiny\color{codegray},
  stringstyle=\color{codepurple},
  basicstyle=\ttfamily\footnotesize,
  breakatwhitespace=false,         
  breaklines=true,                 
  captionpos=b,                    
  keepspaces=true,                 
  numbers=left,                    
  numbersep=5pt,                  
  showspaces=false,                
  showstringspaces=false,
  showtabs=false,                  
  tabsize=2
}

%"mystyle" code listing set
\lstset{style=mystyle}

 
\newcommand{\N}{\mathbb{N}}
\newcommand{\Z}{\mathbb{Z}}
 
\newenvironment{theorem}[2][Theorem]{\begin{trivlist}
\item[\hskip \labelsep {\bfseries #1}\hskip \labelsep {\bfseries #2.}]}{\end{trivlist}}
\newenvironment{lemma}[2][Lemma]{\begin{trivlist}
\item[\hskip \labelsep {\bfseries #1}\hskip \labelsep {\bfseries #2.}]}{\end{trivlist}}
\newenvironment{exercise}[2][Exercise]{\begin{trivlist}
\item[\hskip \labelsep {\bfseries #1}\hskip \labelsep {\bfseries #2.}]}{\end{trivlist}}
\newenvironment{problem}[2][Problem]{\begin{trivlist}
\item[\hskip \labelsep {\bfseries #1}\hskip \labelsep {\bfseries #2.}]}{\end{trivlist}}
\newenvironment{question}[2][Question]{\begin{trivlist}
\item[\hskip \labelsep {\bfseries #1}\hskip \labelsep {\bfseries #2.}]}{\end{trivlist}}
\newenvironment{corollary}[2][Corollary]{\begin{trivlist}
\item[\hskip \labelsep {\bfseries #1}\hskip \labelsep {\bfseries #2.}]}{\end{trivlist}}

\newenvironment{solution}{\begin{proof}[Solution]}{\end{proof}}
 
\begin{document}
 
% --------------------------------------------------------------
%                         Start here
% --------------------------------------------------------------
 
\title{Discussion 6}%replace X with the appropriate number
\author{Mengxiang Jiang\\ %replace with your name
CSEN 5303 Foundations of Computer Science} %if necessary, replace with your course title
 
\maketitle
 
\begin{problem}{1} %You can use theorem, exercise, problem, or question here.  Modify x.yz to be whatever number you are proving
Discuss the data structure \emph{queue} and its fundamental operations. Give an illustrative example.

A \emph{queue} is an ordered group of same objects such that 
the first objects added to the group is also the first objects to leave (First In First Out FIFO). 
The two fundamental operations are \emph{enqueue} and \emph{dequeue}.
\emph{Enqueue} adds new objects to the end of the queue while \emph{dequeue} removes elements from the front of the queue.

An illustrative example is suppose a group of students arrive at the Javelina Dining Hall.
They enqueue into a line with Alice at the front, then Bob, then Charlie, and finally Danielle at the end. 
The receptionist scans Alice's id first and dequeues her, allowing her into the dining hall.
Then she scans Bob and dequeues him. In the meantime, Eve and Frank arrive and enqueue behind Danielle.
The receptionist continues scanning ids and dequeueing from Charlie until Frank and the queue is empty.\\

\begin{center}
Example of the student queue\\
\begin{tabular}{|c c c|} 
 \hline
 Step & Operation & Queue \\
 \hline
 1 & Enqueue Alice & Alice \\ 
 2 & Enqueue Bob & Alice, Bob \\
 3 & Enqueue Charlie & Alice, Bob, Charlie \\
 4 & Enqueue Danielle & Alice, Bob, Charlie, Danielle \\
 5 & Dequeue & Bob, Charlie, Danielle \\
 6 & Dequeue & Charlie, Danielle \\
 7 & Enqueue Eve & Charlie, Danielle, Eve \\
 8 & Enqueue Frank & Charlie, Danielle, Eve, Frank \\
 9 & Dequeue & Danielle, Eve, Frank \\
 10 & Dequeue & Eve, Frank \\
 11 & Dequeue & Frank \\
 12 & Dequeue &  \\
 \hline
\end{tabular}
\end{center}
\end{problem}
\pagebreak
\begin{problem}{3}
    Indicate whether each of the following applications would be suitable for a queue.

    a. An ailing company wants to evaluate employee records so as to lay off some workers
    on the basis of service time(the most recently hired employees are laid off first).\\
    No, since the most recently hired employees will be at the end of a queue while dequeueing happens at the front, a stack is more suitable.\\
 
    b. A program is to keep track of patients as they check into a clinic, assigning them to
    doctors on a first come, first served basis.\\
    Yes, assuming there's no priority for emergency or intensive care, the first come first served nature is perfect for a queue.\\

    c. A program to solve a maze is to backtrack to an earlier position (the last place where a choice was made) when a dead-end position is reached.\\
    No, since the most recent place a choice was made will be at the end of the queue while dequeueing happens at the front (or the oldest decision made), a stack is more suitable.\\

    d. An inventory of parts is to be processed by part number.\\
    Yes, as long as you will enqueue the parts in the order of their part number, then the dequeue for processing will also be in order of part number.\\
    
    e. An operating system is to process requests for computer resources by allocating the resources in the order in which they are requested.\\
    Yes, the first come first served nature allocating resources fits a queue.\\

    f. A grocery chain wants to run a simulation to see how the average customer wait time
    would be affected by changing the number of checkout lines in its stores.\\
    Yes, since each checkout line processes the customer's purchase in the order they arrive, they are queues.\\

    g. A dictionary of words used by a spelling checker is to be initialized.\\
    No, a spelling checker needs to have fast (O(1)) access to any element in its dictionary, while a queue's access is slow (O(n)) for most of its elements.\\

    h. Customers are to take numbers at a bakery and be served in order when their numbers come up.\\
    Yes, since service is in order of first come first serve, it is a queue.\\

    i. Gamblers are to take numbers in the lottery and win if their numbers are picked.\\
    No, the lottery does not award the prize in order of who entered but randomly selects one.

    
\end{problem}
% --------------------------------------------------------------
%     You don't have to mess with anything below this line.
% --------------------------------------------------------------
 
\end{document}