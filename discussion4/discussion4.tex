%%%%%%%%%%%%%%%%%%%%%%%%%%%%%%%%%%%%%%%%%%%%%%%%%%%%%%%%%%%%%%%
%
% Welcome to writeLaTeX --- just edit your LaTeX on the left,
% and we'll compile it for you on the right. If you give
% someone the link to this page, they can edit at the same
% time. See the help menu above for more info. Enjoy!
%
%%%%%%%%%%%%%%%%%%%%%%%%%%%%%%%%%%%%%%%%%%%%%%%%%%%%%%%%%%%%%%%

% --------------------------------------------------------------
% This is all preamble stuff that you don't have to worry about.
% Head down to where it says "Start here"
% --------------------------------------------------------------
 
\documentclass[12pt]{article}
 
\usepackage[margin=1in]{geometry}
\usepackage{amsmath,amsthm,amssymb}
 
\newcommand{\N}{\mathbb{N}}
\newcommand{\Z}{\mathbb{Z}}
 
\newenvironment{theorem}[2][Theorem]{\begin{trivlist}
\item[\hskip \labelsep {\bfseries #1}\hskip \labelsep {\bfseries #2.}]}{\end{trivlist}}
\newenvironment{lemma}[2][Lemma]{\begin{trivlist}
\item[\hskip \labelsep {\bfseries #1}\hskip \labelsep {\bfseries #2.}]}{\end{trivlist}}
\newenvironment{exercise}[2][Exercise]{\begin{trivlist}
\item[\hskip \labelsep {\bfseries #1}\hskip \labelsep {\bfseries #2.}]}{\end{trivlist}}
\newenvironment{problem}[2][Problem]{\begin{trivlist}
\item[\hskip \labelsep {\bfseries #1}\hskip \labelsep {\bfseries #2.}]}{\end{trivlist}}
\newenvironment{question}[2][Question]{\begin{trivlist}
\item[\hskip \labelsep {\bfseries #1}\hskip \labelsep {\bfseries #2.}]}{\end{trivlist}}
\newenvironment{corollary}[2][Corollary]{\begin{trivlist}
\item[\hskip \labelsep {\bfseries #1}\hskip \labelsep {\bfseries #2.}]}{\end{trivlist}}

\newenvironment{solution}{\begin{proof}[Solution]}{\end{proof}}
 
\begin{document}
 
% --------------------------------------------------------------
%                         Start here
% --------------------------------------------------------------
 
\title{Discussion 4}%replace X with the appropriate number
\author{Mengxiang Jiang\\ %replace with your name
CSEN 5303 Foundations of Computer Science} %if necessary, replace with your course title
 
\maketitle
 
\begin{problem}{statement} %You can use theorem, exercise, problem, or question here.  Modify x.yz to be whatever number you are proving
Feel free to answer one of the following questions:

1. Discuss bubble sort algorithm. Give an illustrative example.

2. Discuss selection sort algorithm. Give an illustrative example.

3. Discuss insertion sort algorithm. Give an illustrative example.
\end{problem}
 
\begin{problem}{3}
Insertion sort works by starting from the 2nd element in the array, and comparing it with the 1st,
swapping with the 1st if the 2nd is smaller. 
Then, it checks the 3rd element, and compares with the previous elements in descending order,
swapping with each if 3rd is smaller than it. If we found an element smaller than the 3rd,
we stop because now we know we have inserted the 3rd element into the correct position. 
Thus we can move on to the 4th element and repeat this process and so on.
We are done when we have checked the last element and inserted it in the correct position.\\\\
An illustrative example is shown below:\\
Using an input array a = [3, 2, 5, 1, 4].\\\\
Iteration 1
\begin{center}
    \begin{tabular}{|c | c c c c c|} 
        \hline
        instruction & a[0] & a[1] & a[2] & a[3] & a[4]\\
        \hline
        a[1] $<$ a[0]? & 3 & 2 & 5 & 1 & 4 \\ 
        yes, so swap & 2 & 3 & 5 & 1 & 4 \\
        next iteration & 2 & 3 & 5 & 1 & 4 \\
        \hline
    \end{tabular}
\end{center}
Iteration 2
\begin{center}
    \begin{tabular}{|c | c c c c c|} 
        \hline
        instruction & a[0] & a[1] & a[2] & a[3] & a[4]\\
        \hline
        a[2] $<$ a[1] ? & 2 & 3 & 5 & 1 & 4 \\ 
        no, next iteration & 2 & 3 & 5 & 1 & 4 \\
        \hline
    \end{tabular}
\end{center}
\pagebreak
Iteration 3
\begin{center}
    \begin{tabular}{|c | c c c c c|} 
        \hline
        instruction & a[0] & a[1] & a[2] & a[3] & a[4]\\
        \hline
        a[3] $<$ a[2] ? & 2 & 3 & 5 & 1 & 4 \\ 
        yes, so swap & 2 & 3 & 1 & 5 & 4 \\
        a[2] $<$ a[1] ? & 2 & 3 & 1 & 5 & 4 \\ 
        yes, so swap & 2 & 1 & 3 & 5 & 4 \\
        a[1] $<$ a[0] ? & 2 & 1 & 3 & 5 & 4 \\ 
        yes, so swap & 1 & 2 & 3 & 5 & 4 \\
        next iteration & 1 & 2 & 3 & 5 & 4 \\
        \hline
    \end{tabular}
\end{center}
Iteration 4
\begin{center}
    \begin{tabular}{|c | c c c c c|} 
        \hline
        instruction & a[0] & a[1] & a[2] & a[3] & a[4]\\
        \hline
        a[4] $<$ a[3] ? & 1 & 2 & 3 & 5 & 4 \\
        yes, so swap & 1 & 2 & 3 & 4 & 5 \\
        a[3] $<$ a[2] ? & 1 & 2 & 3 & 4 & 5 \\
        no, we're done & 1 & 2 & 3 & 4 & 5 \\
        \hline
    \end{tabular}
\end{center}


\end{problem}
 
% --------------------------------------------------------------
%     You don't have to mess with anything below this line.
% --------------------------------------------------------------
 
\end{document}