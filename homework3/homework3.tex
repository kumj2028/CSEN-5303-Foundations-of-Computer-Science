%%%%%%%%%%%%%%%%%%%%%%%%%%%%%%%%%%%%%%%%%%%%%%%%%%%%%%%%%%%%%%%
%
% Welcome to writeLaTeX --- just edit your LaTeX on the left,
% and we'll compile it for you on the right. If you give
% someone the link to this page, they can edit at the same
% time. See the help menu above for more info. Enjoy!
%
%%%%%%%%%%%%%%%%%%%%%%%%%%%%%%%%%%%%%%%%%%%%%%%%%%%%%%%%%%%%%%%

% --------------------------------------------------------------
% This is all preamble stuff that you don't have to worry about.
% Head down to where it says "Start here"
% --------------------------------------------------------------
 
\documentclass[12pt]{article}
 
\usepackage[margin=1in]{geometry}
\usepackage{amsmath,amsthm,amssymb}

\usepackage{listings}
\usepackage{xcolor}

%New colors defined below
\definecolor{codegreen}{rgb}{0,0.6,0}
\definecolor{codegray}{rgb}{0.5,0.5,0.5}
\definecolor{codepurple}{rgb}{0.58,0,0.82}
\definecolor{backcolour}{rgb}{0.95,0.95,0.92}

%Code listing style named "mystyle"
\lstdefinestyle{mystyle}{
  backgroundcolor=\color{backcolour}, commentstyle=\color{codegreen},
  keywordstyle=\color{magenta},
  numberstyle=\tiny\color{codegray},
  stringstyle=\color{codepurple},
  basicstyle=\ttfamily\footnotesize,
  breakatwhitespace=false,         
  breaklines=true,                 
  captionpos=b,                    
  keepspaces=true,                 
  numbers=left,                    
  numbersep=5pt,                  
  showspaces=false,                
  showstringspaces=false,
  showtabs=false,                  
  tabsize=2
}

%"mystyle" code listing set
\lstset{style=mystyle}

 
\newcommand{\N}{\mathbb{N}}
\newcommand{\Z}{\mathbb{Z}}
 
\newenvironment{theorem}[2][Theorem]{\begin{trivlist}
\item[\hskip \labelsep {\bfseries #1}\hskip \labelsep {\bfseries #2.}]}{\end{trivlist}}
\newenvironment{lemma}[2][Lemma]{\begin{trivlist}
\item[\hskip \labelsep {\bfseries #1}\hskip \labelsep {\bfseries #2.}]}{\end{trivlist}}
\newenvironment{exercise}[2][Exercise]{\begin{trivlist}
\item[\hskip \labelsep {\bfseries #1}\hskip \labelsep {\bfseries #2.}]}{\end{trivlist}}
\newenvironment{problem}[2][Problem]{\begin{trivlist}
\item[\hskip \labelsep {\bfseries #1}\hskip \labelsep {\bfseries #2.}]}{\end{trivlist}}
\newenvironment{question}[2][Question]{\begin{trivlist}
\item[\hskip \labelsep {\bfseries #1}\hskip \labelsep {\bfseries #2.}]}{\end{trivlist}}
\newenvironment{corollary}[2][Corollary]{\begin{trivlist}
\item[\hskip \labelsep {\bfseries #1}\hskip \labelsep {\bfseries #2.}]}{\end{trivlist}}

\newenvironment{solution}{\begin{proof}[Solution]}{\end{proof}}
 
\begin{document}
 
% --------------------------------------------------------------
%                         Start here
% --------------------------------------------------------------
 
\title{Homework 3}%replace X with the appropriate number
\author{Mengxiang Jiang\\ %replace with your name
CSEN 5303 Foundations of Computer Science} %if necessary, replace with your course title
 
\maketitle
 
\begin{problem}{1} %You can use theorem, exercise, problem, or question here.  Modify x.yz to be whatever number you are proving
\emph{Euclid's algorithm} (or the \emph{Euclidean algorithm}) is an algorithm that computes the \emph{greatest common divisor}, 
denoted by \emph{gcd}, of two integers. Below are the original versions of Euclid's algorithm that uses \emph{repeated subtraction}
and another one that uses the \emph{remainder}.
\begin{verbatim}
    int gcd_sub(int a, int b)               int gcd_rem(int a, int b)
    {                                       {
        if (!a)                                 int t;
            return b;                           while (b)
        while (b)                               {
            if (a > b)                              t = b;
                a = a - b;                          b = a % b;
            else                                    a = t;
                b = b - a;                      }
        return a;                               return a;
    }                                       }
\end{verbatim}
1. Trace each of the above aglorithm using specific values for $a$ and $b$.
\begin{center}
    \begin{tabular}{|c c c|} 
        \hline
        Iteration & $a$ & $b$ \\
        \hline
        1 & 119 & 544 \\ 
        2 & 119 & 425 \\
        3 & 119 & 306 \\
        4 & 119 & 187 \\
        5 & 119 & 68 \\
        6 & 51 & 68 \\
        7 & 51 & 17 \\
        8 & 34 & 17 \\
        9 & 17 & 17 \\
        10 & 17 & 0 \\
        \hline
    \end{tabular}

    Trace with $a = 119$ and $b = 544$ for $gcd\_sub$.

    \begin{tabular}{|c c c|} 
        \hline
        Iteration & $a$ & $b$ \\
        \hline
        1 & 119 & 544 \\ 
        2 & 544 & 119 \\
        3 & 119 & 68 \\
        4 & 68 & 51 \\
        5 & 51 & 17 \\
        6 & 17 & 0 \\
        \hline
    \end{tabular}
    
    Trace with $a = 119$ and $b = 544$ for $gcd\_rem$.
\end{center}
2. Compare both algorithms.\\
Algorithm $gcd\_sub$ is simpler by using only subtraction with no need for a temporary variable. 
However, it will take a lot more iterations when $a$ and $b$ greatly differ in size compared to $gcd\_rem$.
Algorithm $gcd\_rem$ is able to reduce the larger of $a$ and $b$ to be smaller than the smaller of $a$ and $b$
in one iteration by taking advantage of the $modulus(\%)$ operator, but this operator is more complicated than subtraction.

\end{problem}

\begin{problem}{2}
    Given a fixed integer $B  (B\geq 2)$, we demonstrate that any integer $N (N\geq 0)$ 
    can be written in a unique way in the form of the sum $p+1$ terms as follows:
    \begin{center}
        $N = a_0 + a_1 \times B + a^2 \times B^2 + \ldots + a_p \times B^p$
    \end{center}
    where all $a_i$, for $0 \leq a_i \leq B - 1$.
    
    The notation $a_pa_{p-1}\ldots a_0$ is called the $representation$ of $N$ in base $B$. 
    Notice that $a_0$ is the remainder of the $Euclidean$ division of $N$ by $B$. 
    If $Q$ is the $quotient$, $a_1$ is the remainder of the $Euclidean$ division
    of $Q$ by $B$, etc.\\\\
    1. Write an algorithm that generates the representation of $N$ in base $B$.
    \begin{verbatim}
        function BaseB(n: int, b: int) : string;
            var
                rem_temp : int;
                char_temp : character;
                result : string;
            begin
                result := "";
                while n > 0 do
                    begin
                        rem_temp := n mod b;
                        char_temp := CharInBaseB(rem_temp, b);
                        result := char_temp + result;
                        n := n / b;
                    end;
                return result;
            end;
    \end{verbatim}
    2. Compute the time complexity of your algorithm.\\
    $T(n) = T(n/b) + \Theta(1)$, $a = 1$, $b = b$, $f(n) = \Theta(1)$\\
    $\log_b{a} = \log_b{1} = 0$, $n^{\log_b{a}} = n^0 = 1 = \Theta(1)$\\
    Since $f(n)$ is the same size as $n^{\log_b{a}}$, case 2 of the master theorem applies,\\
    so $T(n) = \Theta(\log(n))$


\end{problem}
 
% --------------------------------------------------------------
%     You don't have to mess with anything below this line.
% --------------------------------------------------------------
 
\end{document}