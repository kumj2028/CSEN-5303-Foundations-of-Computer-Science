%%%%%%%%%%%%%%%%%%%%%%%%%%%%%%%%%%%%%%%%%%%%%%%%%%%%%%%%%%%%%%%
%
% Welcome to writeLaTeX --- just edit your LaTeX on the left,
% and we'll compile it for you on the right. If you give
% someone the link to this page, they can edit at the same
% time. See the help menu above for more info. Enjoy!
%
%%%%%%%%%%%%%%%%%%%%%%%%%%%%%%%%%%%%%%%%%%%%%%%%%%%%%%%%%%%%%%%

% --------------------------------------------------------------
% This is all preamble stuff that you don't have to worry about.
% Head down to where it says "Start here"
% --------------------------------------------------------------
 
\documentclass[12pt]{article}
 
\usepackage[margin=1in]{geometry}
\usepackage{amsmath,amsthm,amssymb}
 
\newcommand{\N}{\mathbb{N}}
\newcommand{\Z}{\mathbb{Z}}
 
\newenvironment{theorem}[2][Theorem]{\begin{trivlist}
\item[\hskip \labelsep {\bfseries #1}\hskip \labelsep {\bfseries #2.}]}{\end{trivlist}}
\newenvironment{lemma}[2][Lemma]{\begin{trivlist}
\item[\hskip \labelsep {\bfseries #1}\hskip \labelsep {\bfseries #2.}]}{\end{trivlist}}
\newenvironment{exercise}[2][Exercise]{\begin{trivlist}
\item[\hskip \labelsep {\bfseries #1}\hskip \labelsep {\bfseries #2.}]}{\end{trivlist}}
\newenvironment{problem}[2][Problem]{\begin{trivlist}
\item[\hskip \labelsep {\bfseries #1}\hskip \labelsep {\bfseries #2.}]}{\end{trivlist}}
\newenvironment{question}[2][Question]{\begin{trivlist}
\item[\hskip \labelsep {\bfseries #1}\hskip \labelsep {\bfseries #2.}]}{\end{trivlist}}
\newenvironment{corollary}[2][Corollary]{\begin{trivlist}
\item[\hskip \labelsep {\bfseries #1}\hskip \labelsep {\bfseries #2.}]}{\end{trivlist}}

\newenvironment{solution}{\begin{proof}[Solution]}{\end{proof}}
 
\begin{document}
 
% --------------------------------------------------------------
%                         Start here
% --------------------------------------------------------------
 
\title{Discussion 2}%replace X with the appropriate number
\author{Mengxiang Jiang\\ %replace with your name
CSEN 5303 Foundations of Computer Science} %if necessary, replace with your course title
 
\maketitle
 
\begin{problem}{statement} %You can use theorem, exercise, problem, or question here.  Modify x.yz to be whatever number you are proving
Feel free to answer one of the following questions:

1. Discuss the selection constructs. Give an illustrative example for each one of them.

2. Discuss the repetition constructs. Give an illustrative example for each one of them.
\end{problem}
 
\begin{problem}{1}
A selection construct is a fancy way to say that an instruction is carried out only if a certain condition is met. We discussed two basic types of selection construct.\\\\
The first is the \emph{if then else} construct. An illustrative example is:
\begin{verbatim}
    if weather = rain then
        bring_umbrella := true;
    else
        bring_umbrella := false;
\end{verbatim}
This is equivalent to the statement ``If the weather is raining, then bring an umbrella, else don't bring an umbrella."\\\\
The second is the \emph{switch case} construct. An illustrative example is:
\begin{verbatim}
    case (food) of
        pancake: utensil := fork;
        cereal: utensil := spoon;
        smoothie: utensil : straw;
    else
        utensil := hand;
    end;
\end{verbatim}
This is equivalent to the statement ``Use a fork to eat pancakes, a spoon to eat cereal, a straw to drink smoothies, and your hands for anything else."

 
\end{problem}
 
% --------------------------------------------------------------
%     You don't have to mess with anything below this line.
% --------------------------------------------------------------
 
\end{document}