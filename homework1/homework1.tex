%%%%%%%%%%%%%%%%%%%%%%%%%%%%%%%%%%%%%%%%%%%%%%%%%%%%%%%%%%%%%%%
%
% Welcome to writeLaTeX --- just edit your LaTeX on the left,
% and we'll compile it for you on the right. If you give
% someone the link to this page, they can edit at the same
% time. See the help menu above for more info. Enjoy!
%
%%%%%%%%%%%%%%%%%%%%%%%%%%%%%%%%%%%%%%%%%%%%%%%%%%%%%%%%%%%%%%%

% --------------------------------------------------------------
% This is all preamble stuff that you don't have to worry about.
% Head down to where it says "Start here"
% --------------------------------------------------------------
 
\documentclass[12pt]{article}
 
\usepackage[margin=1in]{geometry}
\usepackage{amsmath,amsthm,amssymb}

\usepackage{listings}
\usepackage{xcolor}

%New colors defined below
\definecolor{codegreen}{rgb}{0,0.6,0}
\definecolor{codegray}{rgb}{0.5,0.5,0.5}
\definecolor{codepurple}{rgb}{0.58,0,0.82}
\definecolor{backcolour}{rgb}{0.95,0.95,0.92}

%Code listing style named "mystyle"
\lstdefinestyle{mystyle}{
  backgroundcolor=\color{backcolour}, commentstyle=\color{codegreen},
  keywordstyle=\color{magenta},
  numberstyle=\tiny\color{codegray},
  stringstyle=\color{codepurple},
  basicstyle=\ttfamily\footnotesize,
  breakatwhitespace=false,         
  breaklines=true,                 
  captionpos=b,                    
  keepspaces=true,                 
  numbers=left,                    
  numbersep=5pt,                  
  showspaces=false,                
  showstringspaces=false,
  showtabs=false,                  
  tabsize=2
}

%"mystyle" code listing set
\lstset{style=mystyle}

 
\newcommand{\N}{\mathbb{N}}
\newcommand{\Z}{\mathbb{Z}}
 
\newenvironment{theorem}[2][Theorem]{\begin{trivlist}
\item[\hskip \labelsep {\bfseries #1}\hskip \labelsep {\bfseries #2.}]}{\end{trivlist}}
\newenvironment{lemma}[2][Lemma]{\begin{trivlist}
\item[\hskip \labelsep {\bfseries #1}\hskip \labelsep {\bfseries #2.}]}{\end{trivlist}}
\newenvironment{exercise}[2][Exercise]{\begin{trivlist}
\item[\hskip \labelsep {\bfseries #1}\hskip \labelsep {\bfseries #2.}]}{\end{trivlist}}
\newenvironment{problem}[2][Problem]{\begin{trivlist}
\item[\hskip \labelsep {\bfseries #1}\hskip \labelsep {\bfseries #2.}]}{\end{trivlist}}
\newenvironment{question}[2][Question]{\begin{trivlist}
\item[\hskip \labelsep {\bfseries #1}\hskip \labelsep {\bfseries #2.}]}{\end{trivlist}}
\newenvironment{corollary}[2][Corollary]{\begin{trivlist}
\item[\hskip \labelsep {\bfseries #1}\hskip \labelsep {\bfseries #2.}]}{\end{trivlist}}

\newenvironment{solution}{\begin{proof}[Solution]}{\end{proof}}
 
\begin{document}
 
% --------------------------------------------------------------
%                         Start here
% --------------------------------------------------------------
 
\title{Homework 1}%replace X with the appropriate number
\author{Mengxiang Jiang\\ %replace with your name
CSEN 5303 Foundations of Computer Science} %if necessary, replace with your course title
 
\maketitle
 
\begin{problem}{1} %You can use theorem, exercise, problem, or question here.  Modify x.yz to be whatever number you are proving
Describe an algorithm that takes as input a list of n integers and produces as output
the largest difference obtained by subtracting an integer in the list from the one following it.
\\\\
\textbf{Algorithm A}. Find largest difference in an integer list $a$ with $n$ elements
\begin{verbatim}
    function LargestDifference(a: arrayInt) : integer;
        var
            max_dif, i, dif : integer;
        begin
            max_dif := INT_MIN;
            for i:=0 to n - 2 do
                begin
                    dif := a[i + 1] - a[i];
                    if max_dif < dif then max_dif := dif;
                end;
            return max_dif;
        end;
\end{verbatim}
Write its corresponding program using your favorite programming language.\\

\begin{lstlisting}[language=Python, caption=Largest Difference]
import math

def find_largest_dif(a):
    n = len(a)
    max_dif = -math.inf
    for i in range(n - 1):
        dif = a[i + 1] - a[i]
        if dif > max_dif:
            max_dif = dif
    return max_dif

\end{lstlisting}

\end{problem}

\begin{problem}{2}
Describe an algorithm that takes as input a list of n integers in non-decreasing order
and produces the list of all values that occur more than once.
\\\\
\textbf{Algorithm B}. Find duplicate integers in a non decreasing integer list $a$ with $n$ elements
\begin{verbatim}
    function Duplicates(a: arrayInt) : stackInt;
        var
            i : integer;
            dup : boolean;
            result : stackInt;
        begin
            dup := false;
            result := new stackInt;
            for i:=0 to n - 2 do
                begin
                    if dup then
                        begin
                            if a[i] <> a[i + 1] then dup = false;
                        end;
                    else
                        begin
                            if a[i] = a[i + 1] then
                            begin
                                dup = true;
                                result.push(a[i]);
                            end;
                        end;
                end;
            return result;
        end;
\end{verbatim}
Write its corresponding program using your favorite programming language.\\

\begin{lstlisting}[language=Python, caption=Duplicate Integers]
def find_duplicates(a):
    n = len(a)
    result = list()
    dup = False
    for i in range(n - 1):
        if dup:
            if a[i] != a[i + 1]:
                dup = False
        else:
            if a[i] == a[i + 1]:
                dup = True
                result.append(a[i])
    return result

\end{lstlisting}

\end{problem}

\begin{problem}{3}
Describe an algorithm that takes as input a list of n integers and finds the total number
of negative integers in the list.
\\\\
\textbf{Algorithm C}. Find number of negative integers in an integer list $a$ with $n$ elements
\begin{verbatim}
    function CountNegatives(a: arrayInt) : integer;
        var
            i, negs : integer;
        begin
            negs := 0;
            for i:=0 to n - 1 do
                if a[i] < 0 then negs := negs + 1;
            return negs;
        end;
\end{verbatim}
Write its corresponding program using your favorite programming language.\\

\begin{lstlisting}[language=Python, caption=Count Negatives]
def count_negatives(il):
    n = len(il)
    negs = 0
    for i in range(n):
        if il[i] < 0:
            negs += 1
    return negs

\end{lstlisting}


\end{problem}
 
% --------------------------------------------------------------
%     You don't have to mess with anything below this line.
% --------------------------------------------------------------
 
\end{document}